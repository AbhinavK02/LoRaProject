\section{Results}
\label{sec: results}
%%%%%%%%%%%%%%%%%%%%%%%%%%%%%%%%%%%%%%%%%%%%%%%%%%%%%%%%%%%%%%%%%%%%%%%%%%%%%%%%%%%%%%%%%%%%%%%%%%%%%%%%%%%%%%%%%%%%%%%%
\subsection{Complete System}
\label{sec: completeSystem}
The completely assembled system can be seen in Figure \ref{fig: finalProduct}. A short demonstration video of the system in action can be found at the following link: \url{https://youtube.com/shorts/1nNW60RcUi4}.
\begin{figure}[ht]
    \begin{subfigure}{0.48\textwidth}
        \centering
        \includegraphics[height=4.5cm]{finalProduct.jpeg}
        \caption{Front View}
        \label{fig: finalProductFront}                   
    \end{subfigure}
    \begin{subfigure}{0.48\textwidth}
        \centering
        \includegraphics[height=4.5cm]{finalProduct_30deg.jpg}
        \caption{Side View}
        \label{fig: finalProduct30deg}                   
    \end{subfigure}
    \caption{Final Product Inside the Post Box}
    \label{fig: finalProduct}
\end{figure}
%%%%%%%%%%%%%%%%%%%%%%%%%%%%%%%%%%%%%%%%%%%%%%%%%%%%%%%%%%%%%%%%%%%%%%%%%%%%%%%%%%%%%%%%%%%%%%%%%%%%%%%%%%%%%%%%%%%%%%%%
\subsection{Issues}
\label{sec: issues}
\subsubsection{TTN Re-join Issue}
\label{sec: ttnRejoinIssue}
The most significant issue faced during the project was with The Things Network (TTN) and the LoRaWAN re-join process. After successfully joining the network for the first time, whenever the device was put to sleep or restarted, it took increasingly longer and longer to re-join the network, sometimes taking upwards of 30 minutes. The error message in the TTN console simply said \cmd{Devnonces too small}. This was a major hindrance to testing and development, as every time the code was modified and re-uploaded, the device had to re-join the network. 

\bigskip

Through extensive research, a number of potential solutions were attempted before the issue was resolved. Firstly, the LoRaWAN specification in the TTN configuration was changed from 1.1.3 to 1.0.4 as this form has more relaxed security requirements. This only resolve the issue for a few power cycles. Next, the RadioLib library was upgraded to the latest version, which required a reworking of most of the LoRaWAN configuration code. This also did not resolve the issue. Finally, it was discovered that the issue was due to the DevNonce value being reset to 0 on every power cycle. This was fixed by storing the DevNonce value in the non-volatile memory of the microcontroller and incrementing it on every join attempt. This solution finally resolved the issue, allowing for quick re-joining on subsequent power cycles as discussed in Section \ref{sec: loraConfig}. 

\subsubsection{Issues with static IP and hosting}
\label{sec: issuesWithStaticIPAndHosting}
The server was initally hosted locally, however to provide access to the user over the web, a public IP address was required. This was also meant to be static, so user can always access the server
without having to worry about changing IP addresses. This is what led to the use of ngrok, which allows for a static IP address to be used. However, this meant the user had to run multiple scripts to start the server. 
Which can be annoying for a non technical user. 

\bigskip

To combat this issue, a .bat file was created, which runs all the necessary commands to start the server and ngrok tunnel. This allows the user to just double click the .bat file and have everything start automatically. This also helps us solve another issue, which is incase the required python libraries are not installed, the script checks 
for them and installs them if they are missing. It also hides all ngrok instances, to prevent confusion. The bat script contents can be seen in Listing \ref{lst: batScript}
\begin{lstlisting}[breaklines=true, caption={.bat Script to start server and ngrok}, label={lst: batScript}]
@echo off
title Smart Mailbox Dashboard
color 0A

echo ========================================================
echo          SMART MAILBOX PROJECT LAUNCHER
echo ========================================================
echo.

:: --- STEP 1: INSTALL/UPDATE REQUIREMENTS ---
echo [1/4] Checking Python libraries...
pip install flask python-dotenv
if %errorlevel% neq 0 (
    color 0C
    echo.
    echo [ERROR] Python or PIP is not installed or not in your PATH.
    echo Please install Python from python.org and try again.
    pause
    exit /b
)
echo Libraries are ready.
echo.

:: --- STEP 2: CHECK CONFIGURATION ---
echo [2/4] Checking configuration file...
if not exist .env (
    color 0E
    echo.
    echo [WARNING] .env file was not found!
    echo I have created a template .env file for you.
    echo Please open ".env", add your passwords, and run this script again.
    
    :: Create a default .env file
    echo SERVER_PORT=3000> .env
    echo AUTH_TOKEN=BC5FB17A739C64639751B59209E07F88>> .env
    echo EMAIL_SENDER=my-iot-project@gmail.com>> .env
    echo EMAIL_PASSWORD=REPLACE_WITH_APP_PASSWORD>> .env
    echo EMAIL_RECIPIENT=your_personal_email@gmail.com>> .env
    
    pause
    exit /b
)
echo Configuration found.
echo.

:: --- STEP 3: START NGROK (NEW WINDOW) ---
echo [3/4] Launching Ngrok Tunnel...
:: This opens a separate popup window for Ngrok so it doesn't block the script
start "Ngrok Tunnel" ngrok http --domain=unarithmetically-peppiest-libbie.ngrok-free.dev 3000

:: --- STEP 4: START PYTHON SERVER ---
echo [4/4] Starting Python Server...
echo.
echo ========================================================
echo    Dashboard: https://unarithmetically-peppiest-libbie.ngrok-free.dev
echo    Local:     http://localhost:3000
echo    Status:    RUNNING (Keep this window open)
echo ========================================================
echo.

python server.py

:: If python crashes, keep window open to see error
pause
\end{lstlisting}

However, this is still not an ideal solution, as the user still has to run the .bat file manually, must have python 
installed on their machine and most importantly run the server continuously. A better solution would be to host the server on a local raspberry pi or similar device, which can run the server 24/7 without any user intervention. This would also eliminate the need for ngrok, as the raspberry pi can be given a static IP address. 
%%%%%%%%%%%%%%%%%%%%%%%%%%%%%%%%%%%%%%%%%%%%%%%%%%%%%%%%%%%%%%%%%%%%%%%%%%%%%%%%%%%%%%%%%%%%%%%%%%%%%%%%%%%%%%%%%%%%%%%%
\subsection{Validation Results}
\label{sec: validationResults}
The results from the validation tests described in Section \ref{sec: validationMethods} are summarized below.

\subsubsection{Mail Detection}
\label{sec: mailDetectionResults}
The system was able to successfully detect 10 out of 10 objects, resulting in a detection accuracy of 100\%. Classification accuracy on the other hand was 80\%, with 8 out of 10 objects being correctly classified into light, medium, and heavy categories. The misclassifications occurred for a stack of papers and a pack of cards. Both objects had weights close to the classification thresholds, leading to incorrect categorization.

\subsubsection{Email Notification}
\label{sec: emailNotificationResults}
The system was able to successfully send email notifications in all 9 test cases. However, events were only classified accurately 7 out of 9 times. The 2 misclassifications were no mail notifications being sent when mail was placed inside. This was due to the weight being read before the mail landed on the platform. To compensate this issue, a delay of 25 seconds was added before reading the weight after mail detection as mentioned in Section \ref{sec: mailDetectionLogic}. It was also found that the entire mail detection event from lid opening to email receipt took approximately 65 seconds on average.

\subsubsection{Battery Life}
\label{sec: batteryLifeResults}
It was found that during an uplink event, the system drew a peak current of 39.5mA on average. While in deep sleep mode, the system drew around 7.25mA on average. This is likely due to the power consumption of the sensors. Using these values, the average current consumption over a 24 hour period can be calculated using the equations below. Assuming 4 mail events per day, each taking 65 seconds, and a battery capacity of 1800mAh, the estimated battery life is approximately 10 days.
\begin{equation}
    I_{awake}^{avg}=\frac{4\cdot39.5mA\cdot65s}{24h\cdot3600\frac{s}{h}}=118.86\mu A
\end{equation}
\begin{equation}
    I_{sleep}^{avg}=\frac{(24h\cdot3600\frac{s}{h}-65s\cdot4)\cdot7.25mA}{24h\cdot3600\frac{s}{h}}=7.23mA
\end{equation}
\begin{equation}
    T=\frac{1800mAh}{7.23mA + 0.11886mA}=244.94 h \approx10days
\end{equation}

Interestingly, the battery life of the system can be very easily extended by using any power bank and a USB-C cable. This is because the XIAO ESP32-S3 has an on board USB-C port which allows for convenient powering. Also, most commercially power banks have a much larger capacity than the 1800mAh battery used in this project. This can be seen in Figure \ref{fig: finalProductWithPowerBank}, where the system is powered by a power bank. Using a power bank with a capacity of 10000mAh, the estimated battery life extends to approximately 55 days.

\begin{figure}[h!]
    \centering
    \includegraphics[width=0.7\textwidth]{finalProduct_wBatteryPack.jpg}
    \caption{Final Product Powered by a Power Bank}
    \label{fig: finalProductWithPowerBank}
\end{figure}
%%%%%%%%%%%%%%%%%%%%%%%%%%%%%%%%%%%%%%%%%%%%%%%%%%%%%%%%%%%%%%%%%%%%%%%%%%%%%%%%%%%%%%%%%%%%%%%%%%%%%%%%%%%%%%%%%%%%%%%%