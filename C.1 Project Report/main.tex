\documentclass[12pt,a4paper]{report}
\usepackage{labreport}

% -------------------------------- Variables ---------------------------------
\renewcommand{\authA}{Justin Julius Chin Cheong}
\renewcommand{\authB}{Abhinav Kothari}
\renewcommand{\matnumA}{34140}
\renewcommand{\matnumB}{33349}
\renewcommand{\project}{The Talking Mailbox}
% --------------------------- Container Templates -----------------------------
% Figure
% \begin{figure}[h!]
%     \centering
%     \includegraphics[width=0.8\textwidth]{smth.png}
%     \caption{brief desc.}
%     \label{fig:smth}
% \end{figure}

% Code
% \begin{lstlisting}[language=C, caption={brief desc.}, label={code:smth}]
% // Code here
% \end{lstlisting}

% Table
% \begin{table}[h!]
%     \centering
%     \begin{tabular}{|c|c|c|}
%         \hline
%         Input 1 & Input 2 & Output \\ \hline
%         0 & 0 & 0 \\ \hline
%         0 & 1 & 1 \\ \hline
%         1 & 0 & 1 \\ \hline
%         1 & 1 & 0 \\ \hline
%     \end{tabular}
%     \caption{desc}
%     \label{tab:smth}
% \end{table}

% \section*{Task :}
% \subsection*{Specifications}
% \subsection*{Circuitry and Logic}
% \subsection*{Function and Results}

%--------------------------------------------------------
%------------Title Section-------------------------------
%--------------------------------------------------------
\begin{document}

% Title
\begin{titlepage}
    \thispagestyle{empty}
    % logo
    \begin{flushright}
        \includegraphics[width=5cm]{hsrw.png}
    \end{flushright}
    
    % title
    \centering
    \vspace{1cm}
    {\Huge \textbf{\project} \par}
    \vspace{1.5cm}
    {\Large 2907 Sensors and Actuator Networks \par}
    \vspace{1 cm}
    {\Large Winter Semester 2025/26 \par}
    \vspace{1.5 cm}

    \vspace{2 cm}

    % author info
    {\Large \textbf{Authors:}}

    \vspace{1 em}

    \begin{tabular}{c c}
    \authA & \authB \\
    \matnumA & \matnumB \\
    MSE & MSE \\
    \end{tabular}
\end{titlepage}

% TOC
\tableofcontents
\thispagestyle{empty}
\clearpage
\pagenumbering{arabic}   % Page number = 1 starts here

\pagebreak

%-------------------------------------------------------------------------
%-----------------Main Content--------------------------------------------
%------------------------------------------------------------------------
\chapter{Introduction}

\section{Problem Statement}

All Professors and Lecturers have a lot to do and may not always have time to check their mailbox. Imagine how long some letters are left in the mailbox for days just because a professor is busy. On the other hand, checking your mailbox only to find nothing is quite frustrating. What if there was a way that your mailbox could tell you when there is mail? What if you had a talking mailbox?

To solve this problem, we introduce \textbf{The Talking Mailbox}. The aim of The Talking Mailbox project is to design and assemble a system that can detect the presence of mail within a mailbox in Building~06 and notify the owner of the mailbox.

\section{Requirements}

\subsection{Functional Requirements}

For The Talking Door to be a satisfiable product, the following functional requirements must be implemented:

\begin{itemize}
    \item It can detect whether or not mail is present within the mailbox.
    \item It can detect if the mailbox is opened.
    \item It can check the battery status.
    \item It can communicate if mail is in the box to a website (based on LoRaWAN).
    \item It can detect light as a redundancy for confirming the opening status of the mailbox.
    \item It alerts the responsible person via email or dashboard upon mail detection.
    \item It sends battery status updates to a website every hour.
    \item It sends a low battery warning to a website when the battery falls below a defined threshold.
\end{itemize}

\subsection{Technical Requirements}

For The Talking Door to operate and perform its functions, the following technical requirements must be implemented:

\begin{itemize}
    \item The weight sensor can detect a change in weight of approximately 20\,g. This indicates when a piece of mail has been placed within the box.
    \item The tilt sensor can detect the rotation of the post box lid. This indicates when the lid is opened.
    \item The LDR can detect the change in light intensity by a defined threshold. This indicates when the lid is opened.
    \item The transmitter can reliably connect and communicate via the LoRaWAN Gateway.
    \item The server with which the LoRaWAN communicates can send emails to relevant personnel about the mail.
    \item The power supply is a battery with a working voltage of 3.1\,V to 5.5\,V.
    \item The enclosure can protect the system within a typical indoor environment (IP\,31).
    \item The system should function at temperatures ranging 0--40\textdegree C and humidity 10--90\%.
\end{itemize}

\subsection{Project Requirements}

For The Talking Mailbox project to produce a functional product upon close out, the following project requirements must be met:

\begin{itemize}
    \item The budget is 100\euro{}.
    \item The project workload is estimated at 100\,h.
    \item The project schedule adheres to the following deadlines:
    \begin{itemize}
        \item Pitch: 2025-10-21
        \item Bill of Materials: 2025-10-23
        \item Schematic Design: 2025-11-23
        \item Project Implementation: 2025-12-19
        \item Project Report: 2026-01-05
        \item Project Presentation and Demo: 2026-01-17
    \end{itemize}
\end{itemize}
%--------------------------------------------------------
% -----------------------------------------------------------
% -----------------------------------------------------------
% -----------------------------------------------------------
% -----------------------------------------------------------
% -----------------------------------------------------------
% -----------------------------------------------------------
% -----------------------------------------------------------
\pagebreak
\appendix
\section*{Appendix}
% -----------------------------------------------------------

\end{document}