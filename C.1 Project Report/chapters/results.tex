\section{Results and Implementation}
\label{sec: resultAndImplementation}
%%%%%%%%%%%%%%%%%%%%%%%%%%%%%%%%%%%%%%%%%%%%%%%%%%%%%%%%%%%%%%%%%%%%%%%%%%%%%%%%%%%%%%%%%%%%%%%%%%%%%%%%%%%%%%%%%%%%%%%%
\subsection{Implementation}
\label{sec: implementation}

\subsubsection{Product Implementation - Hardware Layer}
\label{sec: HW}

\medskip
\textbf{\underline{Final Product Assembly}}
\smallskip

Considering all the design choices made in Section \ref{sec: systemDesign}, the actual implementation of the hardware layer was done as shown in Figure \ref{fig: finalProduct}.
\begin{figure}[ht]
    \begin{subfigure}{0.48\textwidth}
        \centering
        \includegraphics[height=4.5cm]{finalProduct.jpeg}
        \caption{Front View}
        \label{fig: finalProductFront}                   
    \end{subfigure}
    \begin{subfigure}{0.48\textwidth}
        \centering
        \includegraphics[height=4.5cm]{finalProduct_30deg.jpg}
        \caption{Side View}
        \label{fig: finalProduct30deg}                   
    \end{subfigure}
    \caption{Final Product Inside the Post Box}
    \label{fig: finalProduct}
\end{figure}

\medskip
\textbf{\underline{Prototyping}}
\smallskip

However, before this complete assembly was constructed, a smaller prototype was made to test the functioning of the sensors and the LoRa communication. This prototype is shown in Figure \ref{fig: prototype}.

\begin{figure}[ht]
    \begin{subfigure}{0.48\textwidth}
        \centering
        \includegraphics[height=4.5cm]{weight-sensor_construction-1.jpg}
        \caption{Weight Sensor Construction}
        \label{fig: prototypeConstruction}                   
    \end{subfigure}
    \begin{subfigure}{0.48\textwidth}
        \centering
        \includegraphics[height=4.5cm]{weight-sensor_prototype-1.jpg}
        \caption{First Assembled Prototype}
        \label{fig: prototypeAssembled}                   
    \end{subfigure}
    \caption{Prototype of the System for Testing}
    \label{fig: prototype}
\end{figure}

\subsubsection{Product Implementation - Embedded Software Layer}
\label{sec: ESW}
Using the prototype shown in Figure \ref{fig: prototype}, the embedded software layer was implemented. The code was written in C++ using the Arduino IDE and flashed to the ESP32-S3 microcontroller. 

\bigskip

The microcontroller code was structured with modularity in mind and divided into four files: 
\begin{itemize}
    \item \textbf{SensorManager.cpp:} This file contains all the functions related to sensor initialization, reading sensor values, and processing the sensor data.
    \item \textbf{SensorManager.h:} This is the header file for SensorManager.cpp, containing function prototypes and necessary libraries and macros.
    \item \textbf{LoRaConfig.h:} This file handles the LoRaWAN communication, initializing required parameters, including libraries, defining functions for LoRa activation and upload.
    \item \textbf{LR-V1.0.ino:} This is the main file which essentially just runs the functions defined in the preceding files in the specific order required. 
\end{itemize}



\subsubsection{Backend Implementation - Application Layer}
\label{sec: backendImplementation}
This section covers the implementation of the backend server and application layer, which is responsible for receiving data from The Things Network, decoding the payload, storing the data, and providing a user interface for monitoring the mailbox status.
The link with the The Things Network is done using webhooks, which send HTTP POST requests to the python server whenever new data is received from the LoRaWAN device.

\medskip
\textbf{\underline{Initial setup}}
\label{sec: initalSetup}
To set up the backend server, we require three files, which include: 'server.py', '.env', and 'requirements.txt'. Each file servers a specific purpose:
\begin{itemize}
    \item \textbf{server.py:} This is the main server file that contains the Flask application code. It handles incoming requests, decodes the payload, updates the mailbox status, and serves the user interface.
    \item \textbf{.env:} This file contains environment variables such as server port, authentication token, email sender credentials, and recipient email addresses. This allows for easy configuration without hardcoding sensitive information in the code.
    \item \textbf{requirements.txt:} This file lists all the Python libraries required to run the server. 
\end{itemize}
These all need to be placed in the same directory for the server to function correctly. The server can then be started by running the command `python server.py` in the terminal.
The depenendencies, to be added to requirements.txt as shown in Listing \ref{lst: requirementstxt}, can be installed using the command `pip install -r requirements.txt`.
\begin{lstlisting}[caption={requirements.txt file contents}, label={lst: requirementstxt}]
Flask==2.3.3
python-dotenv==1.0.0
\end{lstlisting}
These can be installed directly using pip as well, however using requirements.txt makes it easier and more future proof. These steps can be also be avoided, and the bash script from Section \ref{sec: issuesWithStaticIPAndHosting} be used directly to install the required
libraries if they are missing.

\smallskip

Next a random authentication token must be generated, for Windows 11, the command shown in Listing \ref{lst: generateAuthToken} can be used.
\begin{lstlisting}[language=bash, caption={Generate random auth token command}, label={lst: generateAuthToken}]
$bytes = New-Object byte[] 16; (New-Object System.Security.Cryptography.RNGCryptoServiceProvider).GetBytes($bytes);
[System.BitConverter]::ToString($bytes) -replace '-'
\end{lstlisting}

Now you can open the .env file using a text editor of your choice (here edit by MS was used). The following variables must be
added as shown in Listing \ref{lst: envFileContents}:
\begin{lstlisting}[caption={.env file contents}, label={lst: envFileContents}]
export SERVER_PORT=3000
export AUTH_TOKEN=<token from above>

EMAIL_SENDER=your_email@gmail.com
EMAIL_PASSWORD=your_app_password_here
\end{lstlisting}

The main contents of server.py are explained in the following sections and full code can be found in the project repository. 

\medskip
\textbf{\underline{Payload Decoder}}
\smallskip

\label{sec: payloadDecoder}
The LoRaWAN is able to send the bits to The Things Network. However for these to be actually useful to the user they must be 
decoded and used to represent relevant information for a user, this includes the mail status, the battery and which post box it is. For this first a payload decoder must be made. This is made keeping in mind how bits were encoded in the first place. The decoder can be seen in Listing \ref{lst: payloadDecoder}
\begin{lstlisting}[language=Python, caption={Payload Decoder Function}, label={lst: payloadDecoder}]
def decode_mailbox_data(base64_string):
    try:
        raw_bytes = base64.b64decode(base64_string)
        if len(raw_bytes) < 4:
            return None, "Error: Short Data", "red", None

        # 1. Decode ID (Hex to Int logic)
        try:
            device_id = int(f"{raw_bytes[0]:x}") 
        except:
            device_id = raw_bytes[0]

        state_byte = raw_bytes[1]
        value_id = raw_bytes[2]
        val = raw_bytes[3]

        # 2. Decode Status (Aligned with new JS Decoder)
        if state_byte == 0x04:
            status, color = "Tampering Alert", "red"
        elif state_byte == 0x05:
            status, color = "Heavy Mail", "#004d40" 
        elif state_byte == 0x06:
            status, color = "Medium Mail", "#00897b" 
        elif state_byte == 0x07:
            status, color = "Light Mail", "#4db6ac" 
        elif state_byte == 0x08:
            status, color = "No Mail", "blue"
        else:
            status, color = f"Unknown: {hex(state_byte)}", "orange"

        # 3. Decode Battery (valueID 0x09)
        battery = val if value_id == 0x09 else None

        return device_id, status, color, battery
            
    except Exception as e:
        logger.error(f"Decoding error: {e}")
        return None, "Decoding Failed", "black", None
\end{lstlisting}
This allows us to correctly identify if a heavy, medium or light package was detected. This information can then be used to update the website to represent the appropriate information and also be included in the mail sent to the user.

The key parts of the decoder are explained below:
\begin{itemize}
    \item "state\_byte" is used to determine the mail status, this is done by checking the value of the byte and mapping it to the appropriate status message.
    \item There is return at the start for robustness, in case the payload is too short or empty. 
    \item "value\_id" has been used as a form of future proofing, in case more values are added to the payload in the future.
\end{itemize}
For actual mapping of device ID to device name and email addresses, a dictionary is used as shown in Listing \ref{lst: deviceMapping}.
\begin{lstlisting}[language=Python, caption={Device mapping method}, label={lst: deviceMapping}]
    # --- Device Mapping ---
    DEVICE_NAMES = {
        33: "PostBox SAN" 
    }
    # --- Email Mapping ---
    DEVICE_RECIPIENTS = {
        33: "email1@gmail.com, email2@gmail.com"
    }
\end{lstlisting}
More can be added as just new rows with the similar syntax just a comma at the end of all rows except the last one. 
Multiple email addresses can also be added by separating them with a comma within the double quotes.\\

\medskip
\textbf{\underline{Frontend Implementation - User Interface}}
\smallskip

\label{sec: frontendImplementation}
For the actual server, which the user interacts, with a python server was created. This server uses the Flask framework to create a simple web application that displays the status of the mailbox. 
The server listens for incoming data from The Things Network and updates the mailbox status accordingly. The relevant code snippet is shown in Listing \ref{lst: flaskServer}
\begin{lstlisting}[caption={Flask Server Code Snippet}, label={lst: flaskServer}]
@app.route('/', methods=['GET'])
def show_dashboard():
    d = dashboard_data
    html = """
    <!DOCTYPE html>
    <html>
    <head>
        <title>Mailbox Monitor</title>
        <meta http-equiv="refresh" content="5">
        <style>
            body {  
                font-family: 'Segoe UI', sans-serif; 
                text-align: center; 
                padding: 40px; 
                background-color: #f0f2f5; }
            .card { 
                background: white; 
                padding: 40px; 
                border-radius: 15px; 
                display: inline-block; 
                box-shadow: 0 4px 12px rgba(0,0,0,0.1);                     
                width: 450px; }
            .status-box {   
                font-size: 32px;                         
                font-weight: bold; 
                margin: 20px 0; 
                padding: 20px; 
                color: white; 
                border-radius: 10px; 
                background-color: {{ d.status_color }}; }
            .battery-indicator { 
                font-weight: bold; 
                font-size: 18px; 
                padding: 5px 15px; 
                border-radius: 20px; 
                display: inline-block; 
                color: white; 
                background-color: {{ d.battery_color }}; }
            .meta { 
                color: #888; 
                font-size: 13px; 
                margin-top: 15px; }
            table { 
                width: 100%; 
                border-collapse: 
                collapse; 
                margin-top: 20px; 
                text-align: left; }
            th, td { 
                padding: 10px; 
                border-bottom: 1px 
                solid #eee; 
                font-size: 14px; }
            .dot { 
                height: 10px; 
                width: 10px; 
                border-radius: 50%; 
                display: inline-block; 
                margin-right: 5px; }
        </style>
    </head>
    <body>
        <div class="card">
            <h1>Smart Mailbox</h1>
            <h2 style="color:#666">{{ d.device_name }}</h2> 
            <div class="status-box">{{ d.status_text }}</div>
            <div class="battery-indicator">{{ d.battery_level }}</div>
            <p class="meta">Last Update: {{ d.timestamp }}</p>
            <h3>Recent Activity</h3>
            <table>
                <tr><th>Time</th><th>Event</th><th>Bat</th></tr>
                
                <tr>
                    <td>{{ event.time }}</td> 
                    <td>
                        <span class="dot" 
                            style="background-color: {{ event.color }};">
                        </span>{{ event.status }}</td>
                    <td>{{ event.battery }}</td>
                </tr>
                
            </table>
        </div>
    </body>
    </html>
    """
    return render_template_string(html, d=d)
\end{lstlisting}

\subsubsection{Issues with static IP and hosting}
\label{sec: issuesWithStaticIPAndHosting}
The server was initally hosted locally, however to provide access to the user over the web, a public IP address was required. This was also meant to be static, so user can always access the server
without having to worry about changing IP addresses. This is what led to the use of ngrok, which allows for a static IP address to be used. However, this meant the user had to run multiple scripts to start the server. 
Which can be annoying for a non technical user. 

To combat this issue, a .bat file was created, which runs all the necessary commands to start the server and ngrok tunnel. 
This allows the user to just double click the .bat file and have everything start automatically. This also helps us solve another issue, which is incase the required python libraries are not installed, the script checks 
for them and installs them if they are missing. It also hides all ngrok instances, to prevent confusion. The bat script contents can be seen in Listing \ref{lst: batScript}
\begin{lstlisting}[breaklines=true, caption={.bat Script to start server and ngrok}, label={lst: batScript}]
@echo off
title Smart Mailbox Dashboard
color 0A

echo ========================================================
echo          SMART MAILBOX PROJECT LAUNCHER
echo ========================================================
echo.

:: --- STEP 1: INSTALL/UPDATE REQUIREMENTS ---
echo [1/4] Checking Python libraries...
pip install flask python-dotenv
if %errorlevel% neq 0 (
    color 0C
    echo.
    echo [ERROR] Python or PIP is not installed or not in your PATH.
    echo Please install Python from python.org and try again.
    pause
    exit /b
)
echo Libraries are ready.
echo.

:: --- STEP 2: CHECK CONFIGURATION ---
echo [2/4] Checking configuration file...
if not exist .env (
    color 0E
    echo.
    echo [WARNING] .env file was not found!
    echo I have created a template .env file for you.
    echo Please open ".env", add your passwords, and run this script again.
    
    :: Create a default .env file
    echo SERVER_PORT=3000> .env
    echo AUTH_TOKEN=BC5FB17A739C64639751B59209E07F88>> .env
    echo EMAIL_SENDER=my-iot-project@gmail.com>> .env
    echo EMAIL_PASSWORD=REPLACE_WITH_APP_PASSWORD>> .env
    echo EMAIL_RECIPIENT=your_personal_email@gmail.com>> .env
    
    pause
    exit /b
)
echo Configuration found.
echo.

:: --- STEP 3: START NGROK (NEW WINDOW) ---
echo [3/4] Launching Ngrok Tunnel...
:: This opens a separate popup window for Ngrok so it doesn't block the script
start "Ngrok Tunnel" ngrok http --domain=unarithmetically-peppiest-libbie.ngrok-free.dev 3000

:: --- STEP 4: START PYTHON SERVER ---
echo [4/4] Starting Python Server...
echo.
echo ========================================================
echo    Dashboard: https://unarithmetically-peppiest-libbie.ngrok-free.dev
echo    Local:     http://localhost:3000
echo    Status:    RUNNING (Keep this window open)
echo ========================================================
echo.

python server.py

:: If python crashes, keep window open to see error
pause
\end{lstlisting}
However, this is still not an ideal solution, as the user still has to run the .bat file manually, must have python 
installed on their machine and most importantly run the server continously. A better solution would be to host the server 
on a local raspberry pi or similar device, which can run the server 24/7 without any user intervention. This would also 
eliminate the need for ngrok, as the raspberry pi can be given a static IP address. 
%%%%%%%%%%%%%%%%%%%%%%%%%%%%%%%%%%%%%%%%%%%%%%%%%%%%%%%%%%%%%%%%%%%%%%%%%%%%%%%%%%%%%%%%%%%%%%%%%%%%%%%%%%%%%%%%%%%%%%%%
\subsection{Validation Results}
\label{sec: validationResults}
%%%%%%%%%%%%%%%%%%%%%%%%%%%%%%%%%%%%%%%%%%%%%%%%%%%%%%%%%%%%%%%%%%%%%%%%%%%%%%%%%%%%%%%%%%%%%%%%%%%%%%%%%%%%%%%%%%%%%%%%