\section{Results}
\label{sec: results}
%%%%%%%%%%%%%%%%%%%%%%%%%%%%%%%%%%%%%%%%%%%%%%%%%%%%%%%%%%%%%%%%%%%%%%%%%%%%%%%%%%%%%%%%%%%%%%%%%%%%%%%%%%%%%%%%%%%%%%%%
\subsection{Complete System}
\label{sec: completeSystem}
The completely assembled system can be seen in Figure \ref{fig: finalProduct}. A short demonstration video of the system in action can be found at the following link: \url{https://youtube.com/shorts/1nNW60RcUi4}.
\begin{figure}[ht]
    \begin{subfigure}{0.48\textwidth}
        \centering
        \includegraphics[height=4.5cm]{finalProduct.jpeg}
        \caption{Front View}
        \label{fig: finalProductFront}                   
    \end{subfigure}
    \begin{subfigure}{0.48\textwidth}
        \centering
        \includegraphics[height=4.5cm]{finalProduct_30deg.jpg}
        \caption{Side View}
        \label{fig: finalProduct30deg}                   
    \end{subfigure}
    \caption{Final Product Inside the Post Box}
    \label{fig: finalProduct}
\end{figure}
%%%%%%%%%%%%%%%%%%%%%%%%%%%%%%%%%%%%%%%%%%%%%%%%%%%%%%%%%%%%%%%%%%%%%%%%%%%%%%%%%%%%%%%%%%%%%%%%%%%%%%%%%%%%%%%%%%%%%%%%
\subsection{Issues}
\label{sec: issues}
\subsubsection{Issues with static IP and hosting}
\label{sec: issuesWithStaticIPAndHosting}
The server was initally hosted locally, however to provide access to the user over the web, a public IP address was required. This was also meant to be static, so user can always access the server
without having to worry about changing IP addresses. This is what led to the use of ngrok, which allows for a static IP address to be used. However, this meant the user had to run multiple scripts to start the server. 
Which can be annoying for a non technical user. 

To combat this issue, a .bat file was created, which runs all the necessary commands to start the server and ngrok tunnel. 
This allows the user to just double click the .bat file and have everything start automatically. This also helps us solve another issue, which is incase the required python libraries are not installed, the script checks 
for them and installs them if they are missing. It also hides all ngrok instances, to prevent confusion. The bat script contents can be seen in Listing \ref{lst: batScript}
\begin{lstlisting}[breaklines=true, caption={.bat Script to start server and ngrok}, label={lst: batScript}]
@echo off
title Smart Mailbox Dashboard
color 0A

echo ========================================================
echo          SMART MAILBOX PROJECT LAUNCHER
echo ========================================================
echo.

:: --- STEP 1: INSTALL/UPDATE REQUIREMENTS ---
echo [1/4] Checking Python libraries...
pip install flask python-dotenv
if %errorlevel% neq 0 (
    color 0C
    echo.
    echo [ERROR] Python or PIP is not installed or not in your PATH.
    echo Please install Python from python.org and try again.
    pause
    exit /b
)
echo Libraries are ready.
echo.

:: --- STEP 2: CHECK CONFIGURATION ---
echo [2/4] Checking configuration file...
if not exist .env (
    color 0E
    echo.
    echo [WARNING] .env file was not found!
    echo I have created a template .env file for you.
    echo Please open ".env", add your passwords, and run this script again.
    
    :: Create a default .env file
    echo SERVER_PORT=3000> .env
    echo AUTH_TOKEN=BC5FB17A739C64639751B59209E07F88>> .env
    echo EMAIL_SENDER=my-iot-project@gmail.com>> .env
    echo EMAIL_PASSWORD=REPLACE_WITH_APP_PASSWORD>> .env
    echo EMAIL_RECIPIENT=your_personal_email@gmail.com>> .env
    
    pause
    exit /b
)
echo Configuration found.
echo.

:: --- STEP 3: START NGROK (NEW WINDOW) ---
echo [3/4] Launching Ngrok Tunnel...
:: This opens a separate popup window for Ngrok so it doesn't block the script
start "Ngrok Tunnel" ngrok http --domain=unarithmetically-peppiest-libbie.ngrok-free.dev 3000

:: --- STEP 4: START PYTHON SERVER ---
echo [4/4] Starting Python Server...
echo.
echo ========================================================
echo    Dashboard: https://unarithmetically-peppiest-libbie.ngrok-free.dev
echo    Local:     http://localhost:3000
echo    Status:    RUNNING (Keep this window open)
echo ========================================================
echo.

python server.py

:: If python crashes, keep window open to see error
pause
\end{lstlisting}
However, this is still not an ideal solution, as the user still has to run the .bat file manually, must have python 
installed on their machine and most importantly run the server continously. A better solution would be to host the server 
on a local raspberry pi or similar device, which can run the server 24/7 without any user intervention. This would also 
eliminate the need for ngrok, as the raspberry pi can be given a static IP address. 
%%%%%%%%%%%%%%%%%%%%%%%%%%%%%%%%%%%%%%%%%%%%%%%%%%%%%%%%%%%%%%%%%%%%%%%%%%%%%%%%%%%%%%%%%%%%%%%%%%%%%%%%%%%%%%%%%%%%%%%%
% \subsection{Validation Results}
% \label{sec: validationResults}
%%%%%%%%%%%%%%%%%%%%%%%%%%%%%%%%%%%%%%%%%%%%%%%%%%%%%%%%%%%%%%%%%%%%%%%%%%%%%%%%%%%%%%%%%%%%%%%%%%%%%%%%%%%%%%%%%%%%%%%%